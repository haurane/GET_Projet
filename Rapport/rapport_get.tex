\documentclass[10pt,a4paper]{article}
\usepackage[utf8]{inputenc}
\usepackage[francais]{babel}
\usepackage[T1]{fontenc}
\usepackage{amsmath}
\usepackage{amsfonts}
\usepackage{amssymb}
\usepackage{listings}
\usepackage{pageGarde/HEIG_STY}
\usepackage[hidelinks]{hyperref}

\title{Projet de Gestion d'Entreprise}
\subtitle{Transfusion d'Étérnité}
\author{João Miguel Domingues Pedrosa \\ Stephane Donnet \\ Loïc Haas \\ Nicolas Kobel}

\acro{GET}
\cours{Gestion d'Entreprise}
\date{\today}
\prof{Destine au Prof. Dider Gern}

\begin{document}
\maketitle
\tableofcontents
\newpage
\section{Introduction}
Ce projet s'inscrit dans le cours de Gestion d'Entreprise donné aux étudiants de troisième année du departement TIC.
Dans ce projet nous mettons en oeuvre les techniques et connaissance acquise lors du semestre.

Le cadre de ce projet est une entreprise semi-fictive nommée \textit{Transfusion d'Étérnité}.
Nicolas Kobel, avec une amie Mme Victory Jaques, est entrain d'étudier la création réelle de cette entreprise.
Les produits décrits ci-dessous sont des produits réels de cette entreprise, les chiffres présentés sont des extrapolations des premiers calculs financiers de Mme Jaques et M. Kobel.
Les autres informations ont étés crées pour ce rapport et n'ont pas (encore) étés réalisés.

Les auteurs aimeraient remercier Mme Jaques pour avoir accepter l'utilisation de son travail dans le cadre de ce projet.
\section{L'entreprise}
\subsection{Le Produit}
\subsection{Le Public Cible}
\section{Brevets}
\section{Depot de Marque}
\section{Processus}
\subsection{Fabrication}
\subsection{Vente Internet}
\subsection{Vente aux Redistributeurs}
\section{Finances}
\subsection{Prix de Fabrication}
\subsection{Objectifs Commerciaux}
\subsection{Prix du Produit}
\subsection{Rentabilité}
\subsection{Plan de Trésorerie}
\section{Actions Publicitaires}
\section{Annexes}
\end{document}