\documentclass[10pt,a4paper]{article}
\usepackage[utf8]{inputenc}
\usepackage[francais]{babel}
\usepackage[T1]{fontenc}
\usepackage{amsmath}
\usepackage{amsfonts}
\usepackage{amssymb}
\usepackage{listings}
\usepackage{pageGarde/HEIG_STY}
\usepackage[hidelinks]{hyperref}
\usepackage{lipsum}
\usepackage{fancyhdr} % En tetes / bas de page
\newcommand{\tde}{Transfusion d'Éternité}
 
%marge des pages
\setlength{\textwidth}{16cm}
\setlength{\textheight}{24cm}
\setlength{\oddsidemargin}{0cm}
\setlength{\voffset}{-1.5cm}
\setlength{\headheight}{15pt}


\title{Projet de Gestion d'Entreprise}
\subtitle{\tde{}}
\author{João Miguel Domingues Pedrosa \\ Stephane Donnet \\ Loïc Haas \\ Nicolas Kobel}

\acro{GET}
\cours{Gestion d'Entreprise}
\date{\today}
\prof{Destine au Prof. Dider Gern}


\pagestyle{fancy}
%en-tête
\lhead{Groupe hypocras}
\chead{Rapport GET - Projet Hypocras}
\rhead{\theAcro}
 
%pied-de-page
\lfoot{HEIG-VD}
\cfoot{\today}
\rfoot{\thepage}

\begin{document}
\maketitle
\tableofcontents
\newpage
\section{Introduction}
Ce projet s'inscrit dans le cours de Gestion d'Entreprise donné aux étudiants de troisième année du departement TIC.
Dans ce projet nous mettons en oeuvre les techniques et connaissance acquise lors du semestre.

Le cadre de ce projet est une entreprise semi-fictive nommée \textit{\tde{}}.
Nicolas Kobel, avec une amie Mme Victory Jaques, est entrain d'étudier la création réelle de cette entreprise.
Les produits décrits ci-dessous sont des produits réels de cette entreprise, les chiffres présentés sont des extrapolations des premiers calculs financiers de Mme Jaques et M. Kobel.
Les autres informations ont étés crées pour ce rapport et n'ont pas (encore) étés réalisés.

Les auteurs aimeraient remercier Mme Jaques pour avoir accepter l'utilisation de son travail dans le cadre de ce projet.
\newpage
\section{L'entreprise}
\tde{} est une entreprise de création de boissons.
Elle produit de l'hypocras, une boisson alcolisée à base de vin et d'épices\footnote{\url{https://fr.wikipedia.org/wiki/Hypocras}}.

\tde{} est une SàRL d'après le droit suisse avec siège à Lausanne.
Les deux fondateurs sont associés gérants à parts égales (10 parts de CHF 1000.-- chacun) et ont la signature colléctive à deux.
L'entreprise ne compte pas d'autres employés.

Le produit est déstiné à la vente directe par internet ainsi qu'à la vente à des revendeurs (magasins et bars).
\subsection{Le Produit}
\tde{} propose pour le moment une gamme unique de produits, nommée \textit{Hypocras \tde{}}.
Il s'agit d'une boisson alcolisée à base de vin, miel et épices.

Le produit est décliné en deux variantes, blanc et rouge.
La seule difference est le vin utilisé comme base, les autres ingrédients et le processus de fabrication sont identiques.

La recette de fabrication s'inspire d'une recette d'hypocras antique conservée dans les écrits de Pline l'Ancien.
%Des variantes de cette recettes ont étés téstés par \tde{} et pouront servire comme base pour une gamme differente.
Le lien avec l'antiquité est pour \tde{} un signe de qualité et de durabilité qui sera éxploité dans les actions publicitaires.

\tde{} s'engage à présenter au client la meilleure qualité possible et propose un produit à base d'ingrédients si possible locaux et bio.
Ainsi le vin est fourni par des vignerons de la côte du lac léman cértifiés bio et le miel par des apiculteurs de la région de la gruyère.
Les épices réstantes sont elles aussi cértifiées de production biologique.

Le produit est présenté dans un packaging sobre et classique.
L'hypocras est vendue par bouteilles de 0.25l aux clients sur internet et aux magasins revendeurs.
La taille des bouteilles de revente pour les bars n'a pas encore étée fixée, mais se trouvera dans entre 0.5l et 1l.

Les bouteilles déstinées aux particuliers sont des bouteilles en verre carrés, refermées par un bouchon en liège.
Une étiquette contenant le nom et le logo de \tde{} ainsi que la teneur en alcool se trouve à l'avant de la bouteille.
Une étiquette avec les ingrédiants de base ainsi que leur provenance et des conseil de service se trouve au dos de la bouteille.

Les bouteilles vendues aux particuliers par biais du site internet sont accompagnés à l'envoi d'un feuillet explicant l'histoire de l'hypocras et de \tde{}.
% photo bouteille ici

%\lipsum
\subsection{Le Public Cible}
\tde{} cible un public romand, jeune et interéssé aux boissons "exotiques".
Par son côté médieval l'hypocras attire avant tout les passionnés d'histoire, de reconstitution et du fantastique.
Par conséquant les joueurs de rôle seront la principale cible du produit.

Ce public se retrouve régulièrement dans des conventions ou des événements participatifs, ainsi que sur des canaux de discussions en ligne, la rencontre avec les potentiels clients est facile à faire sur internet ou dans les dites conventions.
C'est aussi un public majoritairement jeune et donc habile avec les nouvelles technologies.
La vente par internet est donc une priorité.
\subsection{Préstations Supplémentaires}
En tant que préstation supplémentaire, \tde{} propose sur son site d'achat la possibilité de personnaliser le packaging.
Le client peut lors de la commande demander un carte avec un message personnalisé et l'envoi sous forme de paquet cadeau.
Cette carte au logo de l'entreprise est imprimé automatiquement dans les lieux de production, l'embalage se fait de manière manuelle.

Une autre possibilité est la commande sur le site avec possibilité de retrait de la marchandise chez un revendeur agréé.
En lieu d'un envoi matériel, le client à la possibilité d'imprimer un bon de retrait avec contrôle numérique éfféctué par le revendeur.
\section{Brevets}
%scan ok
\section{Depot de Marque}
%integerer .doc
\section{Processus}

\subsection{Fabrication}

\subsection{Vente Internet}
% OK integrer
\subsection{Vente aux Redistributeurs}

\section{Finances}

\subsection{Prix de Fabrication}
%reporter
\subsection{Objectifs Commerciaux}
%reporter
\subsection{Prix du Produit}
%reporter
\subsection{Rentabilité}
%reporter
\subsection{Plan de Trésorerie}
%reporter
\section{Actions Publicitaires}
Les actions publicitaires envisagées sont orienté vers la rencontre directe avec nôtre public cible.

Dans ce cadre nous présenterons nos produits lors de conventions de jeux de rôles ainsi que lors de reconstitutions historiques.
Lors de ces présentations nous propseront des dégustations et de la vente directe.
Il est aussi envisageable de faire des démonstrations intléractive du brassage lors de conventions de durée moyenne à longue.

Nous souhaitons aussi rechercher un public plus large, par le biais de dégustations chez des réstaurateurs proposant une grande variété de boissons artisanales.

La campagne de lancement se fera dans la région ciblée, à savoir la suisse romande.
Dans un premier temps des distributions de flyers et de l'affichage se feront à Lausanne.
Une campagne similaire est envisagée dans les autres villes romandes.

La période de noël est vue par \tde{} comme une des deux periodes à forte demande (l'autre étant celle des conventions de jeux de rôles en été).
A cette occasion nous éspèrons atteindre un maximum de clients potentiel avec une grand présence lors des marchés de noël.
Les clients connus receverons lors de cette période des promotions afin de les fidéliser.
\section{Annexes}

\end{document}